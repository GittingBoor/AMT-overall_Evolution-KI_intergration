\section{Fazit}
Automatische Musiktranskription entwickelt sich stetig weiter.
Besonders in den letzten Jahren hat dieses Forschungsgebiet erhebliche Fortschritte gemacht.
Verantwortlich dafür ist vor allem die rasante Entwicklung Künstlicher Intelligenz.
In nur wenigen Jahren wurden unzählige KI-Modelle entwickelt, die im Monatsrhythmus beachtliche Fortschritte zeigten.
Fast jedes Problem, was vorher durch Algorithmen gelöst wurde,
konnte durch ein schnelleres und besseres KI-Modell ersetzt werden.
Dies gilt auch für Algorithmen und Architekturen in AMT-Systemen.
CNNs und RNNs entwickelten sich in kürzester Zeit zum neuen Standard innerhalb der AMT-Forschung.

Dieses Phänomen ist jedoch erst der Anfang von Künstlicher Intelligenz und deren Anwendung in AMT-Systemen.
So schnell wie CNNs und RNNs die automatische Musiktranskription beeinflusst haben,
ebenso schnell treten bereits neue, verbesserte KI-Modelle auf den Plan.
Mit dem Transformer-basierten KI-Modell MT3 wurde erneut ein neuer Stand der Technik erreicht,
der die alten KI-Modelle ersetzten sollte.
Es ist zu erwarten, dass sich dieser Zyklus noch vielfach wiederholen wird,
da wir uns noch am Anfang der KI-Forschung befinden.

Trotz dieser großen Meilensteine gibt es in der automatischen Musiktranskription noch einige offene Probleme,
welche gelöst werden müssen.
Datensätze sind unzureichend, transkribierte Noten sind fehlerhaft oder unvollständig
und rauschende Audioaufnahmen verwirren die KI-Modelle zu stark.
KI-Modelle haben zahlreiche Probleme und Fehler beseitigt,
gleichzeitig öffneten sie neue Fehlerquellen, die zuvor nicht existierten.
Ein Beispiel dafür ist das sogenannte Blackbox-Verhalten moderner KI-Modelle.
Dadurch wird unklarer, was an dem Datensatz oder direkt in dem KI-Modell am besten verändert werden muss,
um ein besseres Ergebnis zu erzielen.

Eines der größten Herausforderungen bleibt jedoch die Überführung in lesbare Notenblätter.
Tatsächlich gibt es einige Tools und Firmen die diese Aufgabe mithilfe eines AMT-System versuchen zu lösen,
häufig sind die daraus resultierenden Notenblätter jedoch kaum brauchbar.
Die Noten werden nicht den richtigen Stimmen zugeordnet und Musikalische regeln werden nicht beachtet.
Das führt dazu, dass die Musiknoten nicht intuitiv spielbar sind.
Eine Lösung dafür wäre ein weiteres KI-Modell,
das sich mithilfe einer MIDI-Datei systematisch damit auseinandersetzt,
wie Musiknoten platziert werden müssen und wann Vorzeichen, Dynamikangaben sowie Artikulationszeichen gesetzt werden.

Im Ganzen ist die Einbindung von Künstlicher Intelligenz ein großer Fortschritt für die automatische Musiktranskription.
In naher Zukunft werden AMT-Systeme kontinuierlich weiterentwickelt
und in nicht allzu langer Zeit könnten diese auch im alltag eingesetzt werden.
