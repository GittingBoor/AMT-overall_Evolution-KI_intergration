\section{Einleitung}

\subsection{Automatic Music Transcription}
Musik ist seit Jahrtausenden ein zentraler Bestandteil unserer Gesellschaft.
Während etwa 2000 bis 0 v.Chr. musikalische Werke meist mündlich überliefert wurden,
entwickelte sich in diesem Zeitraum auch eine Notenschrift.
Diese Notenschrift ermöglichte es, Musikstücke einfacher zu erlernen und einem breiteren Publikum zugänglich zu machen.
Durch die Digitalisierung erhielten Digital Audio Workstations zunehmend Einzug in die Musikproduktion,
wodurch Notenblätter oft nicht mehr notwendig waren und es weniger bedarf gab diese Lieder zu übersetzen in Notenschrift.

An dieser Stelle setzt Automatic Music Transcription (AMT) an.
AMT ist ein Prozess, bei dem eine Audiospur als Input gegeben wird und diese durch Computerprogramme
Notenblätter oder, was weiter verbreitet ist, MIDI-Dateien als Output wiedergeben.
Dabei werden durch mehrere Prozesse die Eigenschaften der Noten, zum Beispiel Frequenz oder Lautstärke,
analysiert und im Kontext des Musikstückes analysiert.

\subsection{Herausforderungen und Hindernisse}
Anstatt das man selber diese Lieder, alleine durchs Gehör,
in Notenschrift überträgt würde diese Aufgabe eine Software für einen erledigen.
Dieses Ziel ist jedoch schwer zu erreichen,
da Musik mehrdimensional ist durch zum Beispiel Zeit, Tonhöhe und Polyphonie.
Vor allem bei polyphonen Musikstücken haben herkömmliche Algorithmen viele Schwierigkeiten.
In diesen fällen müssen sie nämlich viele verschiedene Stimmen gleichzeitig analysieren und
im späteren auch die jeweiligen Töne voneinander differenzieren und eindeutig einem Instrument zuordnen.
Ein weiteres Problem ist die Individualität jedes Musikstückes.
In realen Aufnahmen können leichtes Rauschen,
kleine Spielfehler oder stilistische Mittel wie Vibrato auftreten,
die je nach Interpret unterschiedlich klingen.
Zudem sind die meisten AMT-Modelle auf westliche Tonleiter trainiert.
Dies kann zu Problemen führen wenn man zum Beispiel 
arabische oder indische Musikstücke transkribieren möchte.

\subsection{AMT und Künstliche Intelligenz}
Um diese vielfalt zu bewältigen ist ein neuer, oft genutzter Ansatz,
die nutzung von Künstlicher Intelligenz und Machine Learning.
Im gegensatz zu Algorithmen ist KI flexibler und kann sich besser einstellen auf
kleine Abweichungen in realitätsnahen Audioaufnahmen von Musikstücken.
...

\subsection{praktische Anwendungsfelder und Vorteile von AMT}
AMT kann auch bei vielen anderen Problemen helfen
oder in vielen Bereichen Quality-of-Life-Changes bringen.
Zum einen kann der Musikunterricht spannender und interaktiver gestaltet werden.
Es gibt eine breitere Auswahl von Musikstücken die man den Schülern anbieten kann,
wodurch diese durch individuell angepasste Musikstücke mehr Spaß und Ehrgeiz beim lernen haben könnten.
Zudem kann man die gespielten Musikstücke der Schüler direkt beim Spielen transkribieren und gezielt erkennen,
wo der jeweilige Schüler noch Verbesserungsmöglichkeiten hat.
Grundsätzlich können deutlich mehr Musikstücke transkribiert werden,
wodurch sich große Archive aufbauen lassen.
Ein größeres Interesse an Musik wird geweckt, da Musikstücke von beliebten Serien, Filmen oder Spielen
leichter für deren Musikbegeisterte Zielgruppe zugänglich sind.
Allein dadurch, dass Computerprogramme Musikstücke besser verstehen,
können darauf aufbauend weitere Tools für die Musikproduktion entwickelt werden.
Auch KI würde davon stark profitieren.
KI-generierte Musik würde verbessert werden, da die KI selber ein besseres Verständnis der Musik entwickelt.
Audiobasierte Suchmaschinen könnten gewünschte Musikstücke oder bestimmte Videos präziser finden.
Musik könnte barrierefreier gestaltet werden,
indem gehörlose Menschen sie lesen können und Musiker beim Spielen direktes Feedback erhalten,
ob sie die Noten korrekt gespielt haben.


\subsection{Motivation und Zielsetzung dieser Arbeit} (Noch nicht angefangen!)
Was wird in der Arbeit behandelt, worauf liegt der Fokus (z.B. KI-Methoden für AMT), warum ist das Thema relevant (z.B. für Musiker, KI-Forschung, Musikpädagogik)?

