\section{AMT-Systeme heutzutage}

\subsection{beliebte KI-Module}
\label{subsec:ki_integration}
KI-Systeme haben, in den letzten Jahren stark an beliebtheit gewonnen.
AMT-Systeme bilden da keine Ausnahmen.
Vor allem durch die integration von CNNs und RNNs konnten AMT-Systeme viele Prozesse deutlich verbessern
und neue Errungenschaften in dem Forschungsgebiet erzielen.
Es gibt jedoch auch weitere wichtige KI-Module die in AMT-Systemen häufig genutzt werden
oder zur integration in Planung stehen.
In diesem Kapitel werde ich auf genau diese KI-Module stärker eingehen
und deren Aufgaben in der automatischen Musik transkription weiter erläutern.

\subsubsection{Convolutional Neural Networks}
CNNs sind neuronale Netze, welcher besonders gut räumlich strukturierte Daten analysieren können.
Deshalb werden diese vor allem in der analyse von Bildern genutzt.
Sie können zum Beispiel erkennen, was auf einem Bild genau passiert oder welche Objekte in einem Bild zu erkennen sind.
Auch KIs wie ChatGPT nutzen eine verbesserte Form von CNNs, um Bilder zu analysieren.
Im Fall der Musiktranskription haben wir als Input Bild das Spektrogramm.
Spektrogramme können ähnliche wie zweidimensionale Bilder gehandhabt werden,
da auf diesen auch alle wichtigen Daten des Inputs Audiosignals zu finden sind.
CNNs bestehen aus drei verschiedenen Schichten, % Hier schichten was sagen

Ein CNN funktioniert folgendermaßen.
% Convolution Layer (Faltungsschicht)
CNNs verwenden Filter, welche jeweils einen 3x3 Pixel Eingabebereich des Input Bildes abdecken.
Diese Filter werden über das gesamte Bild gezogen und analysieren dadurch erstmal jeden Teil einzeln.
Ein Skalarprodukt aus Filter und Eingabebereich beschreibt dann einen Aktivierungswert.
Wenn man den Aktivierungswert mit anderen Aktivierungswerten vergleicht,
können so mit Patterns und Eigenschaften erkannt werden.
In der Musiktranskription filtert man somit Onsets, Sustains oder harmonische Verläufe heraus.
Zum Beispiel Onsets werden erkannt, wenn es eine plötzliche Energieänderung gibt.


\subsubsection{Recurrent Neural Networks}

\subsubsection{Long Short-Term Memory}

\subsubsection{Transformers}

\subsubsection{Weitere Deep-Learning-Module und Entwicklungen}


\subsection{KI-basierende AMT-Systeme im Vergleich}
% Einige AMT-Systeme raussuchen, die von KI gestütz werden und vor kurzem (3Jahre) gebaut wurden + deren Aufbau erklären