\section{AMT-Systeme heutzutage}

\subsection{heutige Hindernisse in AMT-Systemen}
Trotz der Einführung von KI-Modulen gibt es noch viele offene Probleme, die noch nicht gelöst werden konnten
oder noch nicht perfekt gelöst sind.
Auch wenn durch die nutzung von KIs viele Fehler geringer oder sogar komplett behoben werden konnten,
agieren diese völlig anders als normale Algorithmen.
Dies führte, zum Teil, auch zu neuen Problemen, welche davor nicht mal bekannt waren.

Die Onset und Offset erkennung von Noten wurde schon ausgiebig in vielen Arbeiten behandelt.
Trotzdem ist diese noch nicht komplett akkurat.
Das liegt daran, dass man On-/Offsets an Lautstärkesprüngen und spektralen Änderungen erkennt.
Bei polyphonen Musikstücken mit vielen verschiedenen Noten kann man jedoch schwieriger diese Unterschiede erkennen.

% Welche Probleme haben auch noch heute KI-basierende AMT-Systeme die noch nicht ganz geregelt sind

\subsection{beliebte KI-Module}
% Welche verschiedenen integrationen von KI gibt es in heutigen AMT-Systemen und welche werden am liebsten verwendet

\subsection{KI-basierende AMT-Systeme im Vergleich}
% Einige AMT-Systeme raussuchen, die von KI gestütz werden und vor kurzem (3Jahre) gebaut wurden + deren Aufbau erklären