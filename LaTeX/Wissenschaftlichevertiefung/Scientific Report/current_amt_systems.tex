\section{KI-basierende AMT-Systeme im Vergleich}
Im Laufe der Forschung zu AMT-Systemen wurden schon einige verschiedenen Architekturen eingesetzt und verbessert.
Jedes neue System hat, unabhängig des KI-Moduls, eine komplett eigene Struktur und vorgehensweise.
Im Verlauf dieser wissenschaftlichen Arbeit habe ich die Geschichte
und viele Konzepte der automatischen Musiktranskription dargestellt.
Deshalb wird sich jetzt dieses Letzte große Kapitel um die state of the art AMT-Systeme handeln.
Dafür stelle ich zwei verschiedene AMT-Systeme vor.
Diese nutzen unterschiedliche KI-Modelle.
Ich werde deren Architektur, sowie deren stärken und schwächen, ausgiebig erläutern.

Zum einen haben wir Omnizart, welches CNNs und RNNs als KI-Modelle nutzt.
\cite{wu2021omnizart}
Je nach Anwendungsfall werden stärker ausgeprägte KI-Modelle genutzt.
Bei der Analyse von Drums werden zum Beispiel nur CNNs verwendet,
während bei Chords und der Melody tiefere CNNs und mehrere RNNs kombiniert werden.
Omnizart ist ein Open-Source-Toolkit für AMT.
Dadurch kann man, je nach Bedarf, ein KI-Modell auswählen was perfekt auf eine bestimmte aufgabe ausgelegt wurde.
Omnizart findet ihren Ursprung im Jahre 2020 am Music and Culture Technology Lab, National Taiwan University.
Omnizart hat seit seiner Gründung keine ausschlaggebenden weiteren Technologien, in der richtung KI, hinzugefügt.
Dafür gibt es viele verschiedene CNNs und RNNs und einen open source code,
welchen man gut zur eigenen Forschung an AMT-Systemen nutzen kann.

Das Transformer-Modell, welches ich näher erläutere, heißt MT3.
Auf dieses bin ich schon ein wenig im vorherigem Kapitel (siehe Transformers → MT3) eingegangen.
Dies liegt daran, dass das MT3-Modell, im gegensatz zu Omnizart, ein alleinstehendes Modell ist,
welches darüber hinaus ein wichtiger Meilenstein in der Transformer-basierten automatischen Musiktranskription bildet.
MT3 ist das erste weit verbreitete Multi-Task-Transkriptionsmodell,
das heißt verschiedene Transkription aufgaben werden von einem einzigen Modell gelöst.
Frühere Modelle brauchten zum Beispiel für Drums und die Melody, wie bei Omnizart ist, verschiedene KI-Modelle.
Durch die Communityversion YourMT3+ wird das KI-Modell, bis heute, immer weiter gepflegt und verbessert.
% Diesen Teil und (siehe Transformers → MT3) verbinden????

\subsection{Omnizart}
\subsection{MT3}

% Einige AMT-Systeme raussuchen, die von KI gestütz werden und vor kurzem (3Jahre) gebaut wurden + deren Aufbau erklären