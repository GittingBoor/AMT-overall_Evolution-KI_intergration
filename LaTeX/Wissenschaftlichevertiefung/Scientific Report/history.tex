\section{Geschichte}

\subsection{Moorer und eines der ersten Musiktranskriptionssysteme}
\subsubsection{Grundlegende Probleme bei AMTs}
Einer der ersten Papers über Automatic Music Transcription wurde von James A. Moorer im Jahr 1977 geschrieben.
\cite{Moorer1977}
In diesem beschreibt Moorer seinen Ansatz, polyphone Musik Audiospuren
direkt über Computerprogramme in Notenschrift zu übertragen.
Während dieses Prozesses fallen ihm schon sehr viele schwierigkeiten auf, die auch in späteren Papern und Arbeiten
eine ausschlaggebende Rolle spielen werden.

Eines dieser Probleme wird von ihm als das \("\)Cocktail-Party-Problem\("\) bezeichnet.
Dieses stellt die Schwierigkeit dar, auf einer Party bestimmten Stimmen zu folgen, während viele verschiedene Stimmen
gleichzeitig erklingen.
Das gleiche Problem liegt in der Noten transkription.
Die meisten Musikstücke haben mehrere Instrumente, welche gleichzeitig spielen.
Öfter gibt es auch Musikstücke wo es für ein Instrument, wie zum Beispiel Violine, mehrere verschiedene Stimmen gibt.
Dies erschwert die Zuordnung bestimmter Noten zu einer gewählten Stimme.
Schon viele Menschen scheitern deshalb daran große Musikstücke richtig zu transkribieren.
Noch schwieriger wird es hier für Computerprogramme.
Anfangs identifizieren diese bestimmte Töne anhand der Frequenz des Tons.
Leider reicht das, wie Moorer feststellt, nicht aus um genaustens zu bestimmen,
welche Töne genau momentan gespielt werden.

Jeder Ton hat Obertöne.
Diese Obertöne sind jeweils das Vielfache von dem Grundton, den man spielt.
Heißt, wenn ich auf einem Klavier den Ton C3 mit 130,81 Hz spiele dann hat dieser
die Obertöne C4 (261,62 Hz), G4 (392,42 Hz) usw.
Wenn man nur C3 spielt erklingen für das Computerprogramm auch die jeweiligen Obertöne,
was man Frequenzüberlagerung nennt.
Diese Frequenzüberlagerung sorgt dafür, das zum Beispiel ein Klavier anders klingt als eine Violine.
Daraus resultiert dann die Klangfarbe (Timbre) eines bestimmten Instrumentes.
% Hier Quelle einfügen die sich mit Frequenzüberlagerung und Klangfarbe auseinandersetzt
Leider konnte Moorer zu dieser Zeit noch nicht die Klangfarbe eines Instruments erkennen.
Er konnte auch noch nicht das Problem der Obertöne lösen,
da die damaligen Algorithmen und Verfahren noch nicht in der Lage waren die Grundfrequenz von den Obertönen zu trennen,
weshalb er sich ausschließlich auf zweistimmige polyphone Musikstücke fokussierte.

Ein weiteres Problem war Rauschen in realistischen Audiospuren und
Stilistische mittel in der Musik, wie zum Beispiel Vibrato.
In real aufgenommenen Audiospuren gibt es immer ein gewisses Hintergrundrauschen.
% Quelle die sich mit Rauschen in Audiospuren befasst
Dieses kann von einem Computerprogramm auch als Note erkannt werden oder verhindern, das bestimmte Noten
richtig vom Computerprogramm erkannt werden.
Moorer hat ein Musikstück analog aufgenommen und dieses dann mit einem 14-bit converter digitalisiert.
Dadurch war das Rauschen nicht weg, aber da er das Musikstück selber aufgenommen hat und dieses konvertiert hat
sorgte es für insgesamt geringeres Rauschen.
Zudem konnte er so die Musiker davon abhalten bestimmte Stilistische mittel zu verwenden,
um bessere Daten zur Transkription zu erhalten.
Stilistische mittel, wie Vibrato, konnten nicht genutzt werden,
da diese eine kleine aber kontinuierliche Veränderung der Frequenz verursachen.
Dadurch kann das Computerprogramm nicht korrekt erkennen, das eigentlich eine einzelne Note gespielt wurde.
Somit war der Onset und Offset der Note komplett falsch.

Das letzte Problem, was Moorer angesprochen hat, ist das Nutzen von nicht
harmonischen Instrumenten wie Trommeln oder einem Schlagzeug.
Diese Instrumente haben keinen eindeutigen Pitch für deren Töne,
sie sind eher abhängig von Rhythms und Lautstärke.
Da Moorers AMT sich jedoch auf das Frequenzmuster der Noten fokussiert,
können diese Musikinstrumente nicht berücksichtigt werden.
% Quelle die eingeht auf Unharmonische Instrumente

\subsubsection{Der Aufbau von Moorers AMT-Systems}

\subsubsection{Umwandlung in Notenschrift}